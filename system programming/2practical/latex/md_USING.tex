Перед запуском необходимо скомпилировать файлы. О том как это сделать я написал в файле \mbox{[}INSTALL.\+md\mbox{]}()

Запуск программы\+:


\begin{DoxyCode}{0}
\DoxyCodeLine{./main}

\end{DoxyCode}


После запуска, программа попросит выбрать режим работы\+:


\begin{DoxyItemize}
\item Честное чтение, честная запись
\item Нечестное чтение, нечестная запись
\end{DoxyItemize}

При выборе 1 варианта программа попросит ввести время работы(в секундах).

Для выхода из программы необходимо использовать сочетание клавиш Ctrl + C.

Пример использования программы(нечестное чтение)\+:


\begin{DoxyCode}{0}
\DoxyCodeLine{1) Честное чтение, честная запись}
\DoxyCodeLine{2) Нечестное чтение, нечестная запись}
\DoxyCodeLine{Выберите действие: 2}
\DoxyCodeLine{Введите количество буферов: 1}
\DoxyCodeLine{Введите количество писателей: 1}
\DoxyCodeLine{Введите количество читателей: 1}
\DoxyCodeLine{Писатель 1 пишет}
\DoxyCodeLine{Писатель 1 создал запись}
\DoxyCodeLine{Читатель 2 читает}
\DoxyCodeLine{Читатель 2 закончил чтение}
\DoxyCodeLine{Писатель 1 пишет}
\DoxyCodeLine{Писатель 1 создал запись}
\DoxyCodeLine{Читатель 2 читает}
\DoxyCodeLine{Читатель 2 закончил чтение}
\DoxyCodeLine{...}
\DoxyCodeLine{\string^C}

\end{DoxyCode}


Пример использования программы(честное чтение)\+:


\begin{DoxyCode}{0}
\DoxyCodeLine{1) Честное чтение, честная запись}
\DoxyCodeLine{2) Нечестное чтение, нечестная запись}
\DoxyCodeLine{Выберите действие: 1}
\DoxyCodeLine{Сколько секунд должна работать программа: 10}
\DoxyCodeLine{Введите количество писателей: 1}
\DoxyCodeLine{Введите количество читателей: 1}
\DoxyCodeLine{Читатель 1 в ожидании}
\DoxyCodeLine{Писатель 2 пишет}
\DoxyCodeLine{Писатель 2 создал запись}
\DoxyCodeLine{Читатель 1 читает}
\DoxyCodeLine{Читатель 1 прочитал следующее: 77}
\DoxyCodeLine{Читатель 1 закончил чтение}
\DoxyCodeLine{Читатель 1 в ожидании}
\DoxyCodeLine{Писатель 2 создал запись}
\DoxyCodeLine{Писатель 2 пишет}
\DoxyCodeLine{Читатель 1 читает}
\DoxyCodeLine{Писатель 2 создал запись}
\DoxyCodeLine{Читатель 1 прочитал следующее: 93}
\DoxyCodeLine{Читатель 1 закончил чтение}
\DoxyCodeLine{...}

\end{DoxyCode}
 